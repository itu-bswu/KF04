\documentclass[11pt,a4paper,titlepage]{article}
\usepackage{textcomp}
\usepackage[latin1]{inputenc}
\usepackage[english]{babel}
\usepackage{graphicx}
\usepackage{fancyhdr}
\DeclareGraphicsExtensions{.pdf,.png,.jpg}

\newcommand{\class}[1]{\textsf{#1}}
\newcommand{\field}[1]{\textit{#1}}

\title{KF04 project report}
\author{
Group 12:\\
Jakob Melnyk\\
Niklas Hansen\\
Emil Juul Jacobsen\\
Jens Dahl M�llerh�j\\
Filip Hjermind Jensen\\
}

\date{May 3, 2011}
\begin{document}
\maketitle
\tableofcontents

\setlength{\parindent}{0pt}
\setlength{\parskip}{1.8ex plus 0.5ex minus 0.2ex}

\clearpage

\section{Features}
\label{Feat}
This chapter describes the features and functions, we have included in our Map
of Denmark project, part 2, Spring 2011.
\begin{itemize}
  \item
  Find shortest route
  \item
  Display path using blue coloured roads
  \item
  Multiple pins
  \item
  Total distance
  \item 
  Estimated travel time 
  \item 
  Bike or car route
  \item 
  Remove a single pin by clicking
  \item 
  Remove all pins
\end{itemize}
\subsection{Find shortest route}
Our map is able to calculate the shortest route between two points. This is done
by clicking the map. Clicking the map places a pin at the clicked location. If
the user clicks on another part of the map, our map will show the shortest route
between those two locations, if there is a route between them.
\subsection{Display path using blue coloured roads}
When the user has selected two or more locations, the map will show the shortest
route highlighted in blue. The algorithm we use for finding this route (A*) is
quite fast, so a user will not feel annoyed by the time it takes to find a
route.
\subsection{Multiple pins}
It is possible for a user to place multiple pins on the map. This is done
similarly to finding the shortest route between two points. When any number of
pins are placed, the user can add an additional one. This will calculate the
route between the latest two pins placed. The route shown will be from pin
1$\rightarrow$ 2$\rightarrow$ 3$\rightarrow$ 4..
\subsection{Total distance}
Our Map of Denmark program calculates the total distance of the route found, a
functionality that assists the end-user in planning their trip. This was
relatively simple to implement, but adds a big functionality.
\subsection{Estimated travel time}
As with the total travel distance, our Map of Denmark calculates the estimated
time it will take to travel the route found by our algorithm. This was
relatively easy to calculate and is a huge benefit for the end user. We estimate
the travel time of bikes to be 15 km/hour and with no breaks. This is
optimistic, but we prefer optimistic to not having the feature for bikes at all.
\subsection{Bike or car route}
It is possible for the user to choose between a car route and a bike route. The
different options will make our algorithm consider what is possible to traverse
for both cars and bikes. It will make sure not to show small paths for cars and
not show highways for bikes.
\subsection{Remove a single pin}
To make it easier to use our Map of Denmark, we have it possible to remove a pin
by clicking on the location it was placed. This makes it possible for the end
user to clear up a missclick without having to shut down the entire Map or
clearing all the pins the user has placed.
\subsection{Remove all pins}
In addition to enabling the user to remove a single pin by clicking on it, we
have also made it possible to remove all the pins currently placed. This can be
done by either pressing the ``c'' key on the keyboard or by clicking the button
on the graphical interface. This means that the user will not have to click
every single point in order to start a new route planning.
\pagebreak
\section{Implementation}
\label{Impl}
This chapter describes our implementation of interesting methods and functions
in our Map of Denmark project, part 2, Spring 2011.
\subsection{Model}
In order to implement the previously mentioned pathfinding the \class{Model}
has been extended to include a list of \class{Edges}. These correspond to all the roads
that are in the current route. Whenever it recieves a new request to find a path
from one \class{Node} to another it uses the static method in the Dijkstra class
and adds the result to the list. When \class{Control} needs to draw the route it
uses a method that returns the entire route as Line objects relative to the view. This approach is identical to what we 
use when the \class{Control} requests all roads that should be drawn as the map
for a given view.

We have also included some statistics of the current route. These are calculated
by the \class{Model} since it's the object that primarliy handles the data. This
information can be calculated in linear time relative to the length of the current route since
the route is saved in memory when first found.
\subsection{Dijkstra}
Originally we inteded to use \class{Dijkstra}s algorithm to find the shortest
path from one node in the graph to another. \class{Dijkstra}s algorithm builds a
\class{Shortest Path Tree} by always looking a the node closest to the source
node. Because of this, whenever the destination node is reached, the shortest
path to that node is found. When that happens we stop the algorithm, because we
have found what we were looking for.

In order to always choose the node closest to the source, we use a
\class{priority queue}, where the distance to the source indexes each node.
Sometimes we want to decrease the index of a node, because we have found a route that is shorter than
the previously indexed route. Because the \class{priority queue} in Java's
library is slow when decreasing keys, we use an implementation from
\cite{algs4} called \class{IndexMinPQ}. Our \class{Dijkstra}s algorithm is a
version inspired from the implementation in \cite{algs4}.

We made some rather big changes to the Dijkstra algorithm to make it fit our
requirements. First of all, we replaced the arrays that are used to hold
\field{edgeTo} and \field{distTo} in the \cite{algs4} with \class{Map}s, so we
can use our objects instead of integers used in \cite{algs4}. 
We have not changed anything in the \class{IndexMinPQ} implementation. It takes
integers as values - this is not a problem, however, because our \class{node}s
have unique index integers.
Another major change we have made compared to the \class{Dijkstra}
implementation from \cite{algs4} is that ours is a static method instead of an
object. This means we do not have to instantiate it before we use it.

To be able to compare the edges differently depending on the type of
transportation the user has selected, we have created an \class{Evaluator}. An
evaluator is given as argument to the pathfinding method and is responsible for
transforming the \class{KrakEdge}s to a value that can be inserted into the
\class{priority queue}.

We have made sure one-way roads are only traversable if you are coming from the
right direction. This check is performed every time a road is going to be added
to the \class{priority queue}. If the road cannot be traversed, it is skipped.

It is important to note that we have found an error in the data from
\textbf{Krak}. Some of the highways are oriented in the wrong direction, so the
pathfinding cannot traverse a long stretch of highway. This will make it look
like it wants the user to get on and off the highway in a very weird way. 

We have included a ``hack'' to avoid this issue. This means that we can always
travel on highways even if they are facing in the opposite direction. This works
and gives accurate routes, because the entry and exit points for the highways
are without error.

This gives accurate roads, but it is a bad fix. We have decided to use it, but
it would be preferable if the data was correct. There are other errors in the
data as well, where roads are not traversable at all, because they are one-way
roads in both directions, which means they cannot be entered from either end.
\subsubsection{A-star (A*)}
As mentioned in the start of the previous section, we originally intended to use
\class{Dijkstra}. After we had implemented it, we learned of another, similar
algorithm called \class{A-star} or \class{A*}. \class{A*} is a variation of
\class{Dijkstra} and, in our case, an improvement in the way it works.

\class{A*} benefits from the fact that the roads are part of an actual map
compared to the abstract graph that Dijkstra works with. This makes it possible
to calculate the fastest possible route from one point to another. This helps us
determine which roads to look at first when creating our \class{Shortest Path
Tree}. Because we look at the ``best'' roads first, we end up looking at fewer
roads than with \class{Dijkstra}.

The smaller the difference between the straight line to the end point and the
fastest possible route, the better \class{A*} works. This makes it a very good
algorithm for calculating the shortest possible route - something that we use
for bike routes. 

It does not work as well for car routes, because it is hard to make a good
heuristic for the straight line to the finish. We have chosen to make an
abstract highway from start to finish where it is possible to travel at 110
km/hour. This will almost always be very, very optimistic and as such,
\class{A*} will look at many roads. It is still very much faster than our
\class{Dijkstra} implementation, so we have decided to use it. It could
certainly be improved, if we found a better heuristic for calculating the
car speed.

In our \class{Evaluator} class, we have created a method that calculates the
heuristic at a given \class{Node} relative to the target \class{Node}. This
enables us to easily add new ways of calculating the heuristic on new vehicle
types.
\subsection{Control}
A lot has changed in \class{Control} since our last version. We have changed a
lot of the methods to work as static methods in seperate \class{Tool} classes,
so \class{Control} will feel less cluttered in this version. 
New logic has taken
the place of the old logic however. Route calculation, pin placement, pin
removal and route display all goes through \class{Control}.

When a user clicks on the map, \class{Control} uses our \class{Tools} classes to
convert the screen coordinates to UTM coordinates. It then stores these in a
\class{List}. 
If the list holds two or more elements, it asks the \class{Model}
to calculate the shortest path between the first and second element, the second
and third element and so on, if any additional elements exist beyond the first
two. 
Every time \class{Control} calls it's internal repaint method, it clears
all the pins currently in the \class{View}. It then recalculates the pin
position(s) according to the new size and postion of the \class{View}.

When a user takes advantage of one of the options to clear all the pins
currently on the Map of Denmark, \class{Control} empties the collection holding
all the pin points, asks \class{Model} to clear the route and asks \class{View}
to repaint.

If a user removes a single pin by clicking on the location of the pin, we run
through our collection and remove one of the pins within a specific range of
pixels. It then makes \class{Model} recalculate the route for all the pins. This
is not an effective implementation - if there are a lot of pins, then removing
one might make the recalculating take a while. We decided to use this
implementation, for now at least, because we wanted a working functionality
that we could later improve upon if nessecary.
\subsection{Car/Bike routes}
Another useful feature we have included is the option to switch between multiple
types of vehicles for pathfinding. We have named these two modes Car and
Bike. In the Car mode the pathfinding will priorize lower travel times, since we
believe this to be most important when driving. Additionally the Car mode will
exclude small paths and pedestrian zones.
The Bike mode prioritizes lower distances highest since intersections and speed-limits
are of relatively low importance here. When using Bike mode the route will not include
highways.

As described earlier these different ways of prioritizing the roads are achieved by
using the \class{Evalutator} that is provided when asking the pathfinding
method for a best path. We have written these two evaluators as static objects
on the \class{Evaluator} class for easy use in the code, but the use of a
seperate object for the evaluations open up for later variation with little
impact on the original code.
\subsection{View}
When looking at the changes in \class{View} relatively little has happened since
most of the new code is located "behind the curtain". The \class{Canvas} has
been extended to allow adding and drawing of the route. This is done in exactly
the same way as with the roads in the map itself. The \class{Canvas} also
includes some points that are positions where the pins are placed. These are
drawn on top of the map and (like the map) needs provided again if the map moves.

When looking at the menu to the left there are a few more options to the user.
There is a new button to remove all pins from the map and two
\class{RadioButtons} to toggle between Car mode and Bike mode in the
pathfinding.

Below the tools for changing the vehicle type, we display information about the
route (and includes spare space for more). This information is changed by a
simple public method in the View and the units will be adjusted accordingly.
\subsection{Performance improvement}
During the last month we have spent a lot of time improving the speed of our
map. We have fiddled with lowering RAM-usage, serialization, using threads,
drawing less roads and more. Below are the interesting and most effective ways
we have sped up performance.
\subsubsection{Serialization}
Everytime we start the program, we run through the entire dataset given to us by
Krak. This data is huge, and because of this it takes quite a lot of time to
start the program. Because we need to load all this data, the user is presented
with a blank screen for a long time, before all these data are loaded and the
program starts.

We started looking for a way to speed up the loading process, so the user has a
map in front of him or her quickly, when the user starts the program. What takes
the most time is running through the data and creating the needed datastructures
(quadtrees, etc.), so if we could skip these steps or speed them up, we
could save a lot of time.

This is where serialization comes into play. By serializing an object, you
transform your object into something that can be passed around, through streams
and so on. So by serializing objects, you can save them to files. If the object
to be serialized contains references to other objects, these will also be
serialized (if they are \class{Serializable} / implements java.io.Serializable).
By doing this, we only need to build our datastructures the first time the
program starts. After the objects have been created, they are been serialized
and saved to files. The next time the user starts the program, we check whether the data has been changed. If it hasn't been changed, we load the objects that we
saved, instead of making them all over.
\subsubsection{Threading}
By serializing our main objects, we cut several seconds of our load times. But
we could do it even faster. We are using several objects, but only few of them
are needed right from the start. So what we can do to speed it up even more, is
loading the few neccesary objects, and then load the rest in the background. We
do this with threads. We load the few objects we need from the start, then
create a new thread to load the rest of the objects, and in the mean time, we
create the window and draw the map.

The same goes for the first run. The user doesn't need to wait for the program
to finish serializing and saving to files. By using threads, we can create the
datastructures, and then immediately show the window to the user, while saving
the objects to files in the background.
\pagebreak
\begin{thebibliography}{99}
\bibitem{algs4}
Robert Sedgewick and Kevin Wayne. \emph {Algorithms, Fourth Edition}. Preliminary Edition Fall 2010. Addision-Wesley 2010.
\end{thebibliography}
\end{document}