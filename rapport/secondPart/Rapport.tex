\documentclass[11pt,a4paper,titlepage]{article}
\usepackage{textcomp}
\usepackage[latin1]{inputenc}
\usepackage[danish]{babel}
\usepackage{graphicx}
\usepackage{fancyhdr}
\DeclareGraphicsExtensions{.pdf,.png,.jpg}

\newcommand{\class}[1]{\textbf{#1}}
\newcommand{\field}[1]{\textit{#1}}

\title{KF04 project report}
\author{
Group 12:\\
Jakob Melnyk\\
Niklas Hansen\\
Emil Juul Jacobsen\\
Filip Hjermind Jensen\\
Jens Dahl M�llerh�j\\
}

\date{May 3, 2011}
\begin{document}
\maketitle
\tableofcontents

\setlength{\parindent}{0pt}
\setlength{\parskip}{1.8ex plus 0.5ex minus 0.2ex}

\clearpage

\section{Features}
\label{Feat}
This chapter describes the features and functions, we have included in our Map
of Denmark project, part 2, Spring 2011.
\begin{itemize}
  \item
  Find shortest path
  \item
  Display path using blue highlighting
  \item
  Multiple pins
  \item
  Total distance
  \item 
  Estimated travel time 
  \item 
  Bike or car route
  \item 
  Remove a single pin by clicking
  \item 
  Remove all pins by using GUI button
\end{itemize}
\subsection{Find shortest route}
Our map is able to calculate the shortest route between two points. This is done
by clicking the map. Clicking the map places a pin at the clicked location. If
the user clicks on another part of the map, our map will show the shortest route
between those two locations, if possible.
\subsection{Display path using blue highlighting}
When the user has selected two or more locations, the map will show the shortest
route highlighted in blue. The algorithm we use for finding this route is pretty
fast, so a user will not feel annoyed by how long it takes.
\subsection{Multiple pins}
It is possible for a user to place multiple pins on the map. This is done in
similarly to finding the shortest route between two points. When any number of
pins are placed, the user can add additional ones. This will calculate the route
between the latest two pins placed. The route shown will be from pin
1->2->3->4..
\subsection{Total distance}
Our Map of Denmark program calculates the total distance of the route found, a
functionality that assists the end-user in planning their trip. This was
relatively simple to implement, but adds a big functionality.
\subsection{Estimated travel time}
As with the total travel distance, our Map of Denmark calculates the estimated
time it will take to travel the route found by our algorithm. This was
relatively easy to calculate and is a huge benefit for the end user.
\subsection{Bike or car route}
It is possible for the user to choose between a car route and a bike route. The
different options will make our algorithm consider what is possible to traverse
for both cars and bikes. It will make sure not to show small paths for cars and
not show highways for bikes.
\subsection{Remove a single pin}
To make it easier to use our Map of Denmark, we have it possible to remove a pin
by clicking on the location it was placed. This makes it possible for the end
user to clear up a tiny missclick without having to shut down the entire Map or
clearing all the pins the user has placed.
\subsection{Remove all pins}
In addition to enabling the user to remove a single pin by clicking on it, we
have also made it possible to remove all the pins currently placed. This can be
done by either pressing the ``c'' key on the keyboard or by clicking the button
on the graphical interface. This avoids the chore of clicking every single point
in order to start a new route planning.
\pagebreak
\section{Implementation}
\label{Impl}
This chapter describes our implementation of interesting methods and functions
in our Map of Denmark project, part 2, Spring 2011.

\subsection{Model}
In order to implement the mentioned pathfinding the Model has been extended to 
include a list of Edges. These correspond to all the roads that are in the current
route. Whenever it recieves a new request to find a path from one Node to another
it uses the static method in the Dijkstra class and adds the result to the list.
When the Control needs to draw the route it uses a method that returns the entire 
route as Line objects relative to the view. This approach is identical to what we 
use when the Control requests all roads that should be drawn as the map for a given 
view.

We have also included some statistics of the current route. These are calculated
by the Model since it's the object that primarliy handles the data. This information
can be calculated in linear time relative to the length of the current route since
the route is saved in memory when first found.

\subsection{Dijkstra}
1. Skrevet ud fra implementationen fra Algorithms men modificeret meget til vores benyttelse. IndexMinPQ er svagt modificeret fra Algorithms.
2. Beskrivelse generelt af Dijkstra.
2. �ndringer: KrakNode istedet for integers (dog bliver index stadig brugt i PQ), Statisk metode istedet for Objekt, Evaluator s� man kan benytte forskellige prioriteringer, Tjek for ensrettet.

\subsection{Control}

\subsection{Car/Bike routes}
Another useful feature we have included is the ability to switch between multiple
modes of pathfinding. We have named these two modes Car and Bike. In the Car mode
the pathfinding will priorize lower travel times, since we believe this to be most
important when driving. Additionally the Car mode will exclude small paths and 
pedestrian zones.
The Bike mode prioritizes lower distances highest since intersections and speed-limits
are of relatively low importance here. When using Bike mode the route will not include
highways.

As described earlier these different ways of prioritizing the roads are achieved by
using the Evalutator that is provided when asking the pathfinding method for a best path.
We have written these two evaluators as static objects on the Evaluator class for easy
use in the code, but the use of a seperate object for the evaluations open up for later
variation with little impact on the original code.

\subsection{View}
When looking at the changes in View relatively little has happened since most of the new
code is located "behind the curtain". The Canvas has been extended to allow adding and
drawing of the route. This is however, as mentioned earlier, done in exactly the same way
as with the roads in the map itself. The Canvas also includes some points that is positions
where the Pins are placed. These are drawn on top of the map and (like the map) needs
provided again if the map moves.

When looking at the menu to the left there are a few more options to the user. There is a
new button to remove all Pins from the map and two RadioButtons to toggle between Car mode
and Bike mode in the pathfinding.

Beneath the tools for pathfinding there are displayed two pieces of information (and
spare space for more). This information is changed by a simple public method in the View
and the units will adjust will be adjusted accordingly.

\pagebreak
\end{document}