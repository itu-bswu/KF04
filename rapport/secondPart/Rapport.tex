\documentclass[11pt,a4paper,titlepage]{article}
\usepackage{textcomp}
\usepackage[latin1]{inputenc}
\usepackage[danish]{babel}
\usepackage{graphicx}
\usepackage{fancyhdr}
\DeclareGraphicsExtensions{.pdf,.png,.jpg}

\newcommand{\class}[1]{\textbf{#1}}
\newcommand{\field}[1]{\textit{#1}}

\title{KF04 project report}
\author{
Group 12:\\
Jakob Melnyk\\
Niklas Hansen\\
Emil Juul Jacobsen\\
Filip Hjermind Jensen\\
Jens Dahl M�llerh�j\\
}

\date{May 3, 2011}
\begin{document}
\maketitle
\tableofcontents

\setlength{\parindent}{0pt}
\setlength{\parskip}{1.8ex plus 0.5ex minus 0.2ex}

\clearpage

\section{Features}
\label{Feat}
This chapter describes the features and functions, we have included in our Map
of Denmark project, part 2, Spring 2011.
\begin{itemize}
  \item
  Find shortest path
  \item
  Display path using blue highlighting
  \item
  Multiple pins
  \item
  Total distance
  \item 
  Estimated travel time 
  \item 
  Bike or car route
  \item 
  Remove a single pin by clicking
  \item 
  Remove all pins by using GUI button
\end{itemize}
\subsection{Find shortest route}
Our map is able to calculate the shortest route between two points. This is done
by clicking the map. Clicking the map places a pin at the clicked location. If
the user clicks on another part of the map, our map will show the shortest route
between those two locations, if possible.
\subsection{Display path using blue highlighting}
When the user has selected two or more locations, the map will show the shortest
route highlighted in blue. The algorithm we use for finding this route is pretty
fast, so a user will not feel annoyed by how long it takes.
\subsection{Multiple pins}
It is possible for a user to place multiple pins on the map. This is done in
similarly to finding the shortest route between two points. When any number of
pins are placed, the user can add additional ones. This will calculate the route
between the latest two pins placed. The route shown will be from pin
1->2->3->4..
\subsection{Total distance}
Our Map of Denmark program calculates the total distance of the route found, a
functionality that assists the end-user in planning their trip. This was
relatively simple to implement, but adds a big functionality.
\subsection{Estimated travel time}
As with the total travel distance, our Map of Denmark calculates the estimated
time it will take to travel the route found by our algorithm. This was
relatively easy to calculate and is a huge benefit for the end user.
\subsection{Bike or car route}
It is possible for the user to choose between a car route and a bike route. The
different options will make our algorithm consider what is possible to traverse
for both cars and bikes. It will make sure not to show small paths for cars and
not show highways for bikes.
\subsection{Remove a single pin}
To make it easier to use our Map of Denmark, we have it possible to remove a pin
by clicking on the location it was placed. This makes it possible for the end
user to clear up a tiny missclick without having to shut down the entire Map or
clearing all the pins the user has placed.
\subsection{Remove all pins}
In addition to enabling the user to remove a single pin by clicking on it, we
have also made it possible to remove all the pins currently placed. This can be
done by either pressing the ``c'' key on the keyboard or by clicking the button
on the graphical interface. This avoids the chore of clicking every single point
in order to start a new route planning.
\pagebreak
\section{Implementation}
\label{Impl}
This chapter describes our implementation of interesting methods and functions
in our Map of Denmark project, part 2, Spring 2011.

\subsection{Model}
1. Hvordan model bruger findPath og gemmer det (og clearer)
2. Metoder til at udregne samlet afstand og k�retid.

\subsection{Dijkstra}
1. Skrevet ud fra implementationen fra Algorithms men modificeret meget til vores benyttelse. IndexMinPQ er svagt modificeret fra Algorithms.
2. Beskrivelse generelt af Dijkstra.
2. �ndringer: KrakNode istedet for integers (dog bliver index stadig brugt i PQ), Statisk metode istedet for Objekt, Evaluator s� man kan benytte forskellige prioriteringer, Tjek for ensrettet.

\subsection{Control}

\subsection{Car/Bike routes}

\subsection{View}
1. Tegning af pins og ekstra route. (Den ekstra route foreg�r p� samme m�de som selve kortet, men bliver tegnet til sidst).

\pagebreak
\end{document}