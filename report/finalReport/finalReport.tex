\documentclass{report}

\usepackage[latin1]{inputenc}
\usepackage[english]{babel}
%\usepackage{fullpage}
\usepackage{graphicx}
\usepackage{longtable}
\usepackage{mathtools}
\usepackage{appendix}
\usepackage{nameref}
\usepackage{fancyhdr}
\usepackage{mathtools}
\usepackage[final]{pdfpages}

\pagestyle{fancy}
%\cfoot{\thepage\ of ??}

\setlength{\parindent}{0pt}
\setlength{\parskip}{1.8ex plus 0.5ex minus 0.2ex}
\newcommand{\class}[1]{\textsf{#1}}
\renewcommand{\chaptermark}[1]{\markboth{\MakeUppercase{\thechapter.\ #1}}{}}

\title{Map of Denmark\\
\normalsize{\textit{First-Year Project, Bachelor in Software Development, \\IT
Univ. of Copenhagen}}}
\author{
Group 12\\
Jakob Melnyk jmel@itu.dk\\
Niklas Hansen nikl@itu.dk\\
Emil Juul Jacobsen ejuu@itu.dk\\
Jens Dahl M�llerh�j jdmo@itu.dk\\
\\
Supervisor: Lars Birkedal\\
Advisors: Jonas Brabrand
Jensen and Filip Sieczkowski
}
\date{May 25th, 2011}

\begin{document}
\maketitle
\tableofcontents
\pagebreak
\chapter{Preface}
\label{PRE}
This report is the result of a project at the bachelor in Software Development 
on the IT University of Copenhagen spanning from early March 2011 to late May
2011. The project was given as a first semester probject in groups of 4 to 5
randomly picked students. 2,5 months of the semester was set aside for the
project. The time not spent on the report was used on coding, thorough testing
and bugfixing of the actual program.

\chapter{Background}
\section{Problem area}

\section{Our requirements}
\subsection{Project requirements}
\subsection{Our own requirements}

\section{Data set}
\subsection{UTM-coordinates}
\subsection{Graph}

\section{MVC structure}
\chapter{User Interface analysis}
\label{UIA}
In this chapter we describe our decisions and present our analysis and
arguments regarding possible features that we find might have been interesting
to have implemented in our \class{Map Of Denmark} program.

\section{User interface as a whole}
\label{UIA-UIW}
When we designed the first version of the graphical user interface in the first
part of the project, we decided to make a window inside of the graphical user
interface where the actual map should be displayed. We chose to have this
window placed on the right side of our graphical user interface and interaction
with the user mainly placed on the left.

We believe that this is a simple way of representing a user interface for a map.
A lot of software use a menu bar with dropdown menus for selecting different
functions. When we designed our outline for the graphical user interface, we did
not design it with a huge amount of functions in mind. 

The features that we have implemented in this version can easily fit in our
simple user interface, but if features like searching for roads or other
features are included, then space and overview may become an issue on the left side.

If new features are included, we feel it would be beneficial to let the main
window change when different feature types are selected.

Below is a screenshot of our user interface. How to use it will be
explained in the \class{\nameref{MAN}} on page \pageref{MAN}.

\begin{figure}[!ht]
\centering
\includegraphics[width=1\linewidth]{images/PictureOfUI}
\caption{Screenshot of GUI}
\label{UIA-UIW-PIC}
\end{figure}

\section{Interesting features}
\label{UIA-IF}
This section presents some of the interesting features we have implemented.
\subsection{Zoom}
\label{UIA-IF-Z}
We have a few options for zooming in and out on the map. As described in
section \ref{BG-R} \class{\nameref{BG-R}} on page \pageref{BG-R}, it
was required that we made it possible to zoom by dragging a box around the part 
of the map the user wants to look closer at.

In addition to the option of using the mouse to zoom, we have implemented a
zoom-in and out function on the GUI and a hotkey for zooming out to the original
zoom level. We made the original zoom function a hotkey only because we did not
want to have too many buttons on the left side. We considered making it a menu bar
item, but we did not manage to get it into this version.

We felt we really needed a zoom out function, so users do not need to close the
program and start it again, when the user wants to view the map further zoomed
out. A combination of the zoom in and out functions helps the user a lot when
navigating the map.

We have limited how far a user can zoom in and out. If the user tries to zoom
out further than the original zoom level, the view will default to the original
zoom level. If the user attempts to zoom in further than a width or height of
200 in UTM32-coordinates, the zoom function will do nothing. This could 
probably be improved by zooming in on the smallest possible area at the 
position the user selected, instead of doing nothing.

\subsection{Navigation}
\label{UIA-IF-N}
We have made it possible for the user to navigate the map by using the arrow
buttons on the graphical user interface. When one of the buttons are pressed,
the ``view'' will move in the direction specified by the button. While it was
not specified as a requirement for the project, we felt it was a necessity to
implement at least basic navigation functionality.

Like we did with the two zoom functionalities, we have limited how far a user
can move around the map. The user is free to move around the map, but if user
moves outside the bounds of the map in a way where the view would show an image
that is not part of the map, the move function will not do anything.

\subsection{Hotkeys}
\label{UIA-IF-H}
We have implemented hotkeys for all the buttons on the graphical user interface
plus an additional for zooming back to the original zoom level. When we
discussed the benefits of hotkeys, we felt it was important for experienced users of the
software should have a less cumbersome time navigating the map. 

At first we just had hotkeys for the clearing of markers (mentioned in section
\ref{UIA-IF-M}) and zooming out to the original zoom level, but we later added
the hotkeys for the rest of the functionalities. If more features are added in a
future version, it would be important for us that a hotkey were provided, if at
all possible.

\subsection{Route planning and markers}
\label{UIA-IF-M}
Part of the requirements for the project was to provide the user with a way to
get the fastest or shortest route from one point to another. We accomplish this
by putting a ``marker'' at the spot where the user clicks with the mouse. The
marker shows which number in the sequence of markers it is. This will change 
if a marker is removed. Originally we had ``pins'' instead of markers, but we 
changed it, as we felt the pins we had were a bit large.

We have made it possible to place more than the two markers the project
requirements asks for. If the user places more than two markers, the software
will find the shortest route between 1 -> 2 and 2 -> 3 and so on. This was cheap for
us to implement, and we felt it added a nice touch to our program. 

We have implemented two methods of removing pins from the map. We have assigned
a hotkey to the graphical user interface button ``Clear Markers'', which removes 
all the markers from the map. The other way of removing pins is by clicking on
them. This functionality is both intuitive and confusing at the same time. It is
intuitive to click the marker you have just placed if you want to remove it, but
it is not obvious in our interface. We believe that it is enough to have the
``clear all markers'' functionality for those who do not find it intuitive to
click markers to remove them, and for the users that do find it intuitive, we
offer them an easy way to undo a missclick.

\subsection{Bike/car}
\label{UIA-IF-BC}
Another interesting feature in our \class{Map of Denmark} project is the option
to switch between bike and car routes. The user interface will start with car
selected when the program starts.

Whenever a route is calculated, it will display the length and the estimated
travel time on the left part of the user interface. When the bike option is selected, 
it recalculates the route for a bike, without visiting highways and other roads 
that bikes cannot or are not allowed to drive on. If the user switches back to the 
car mode, it recalculates the route again, but not visiting small paths and other 
roads where a car is not allowed to drive. The estimated travel time is also 
recalculated. The user does not need to have planned a route before he/she 
changes the type of transportation.

We have implemented this to help our software target a wider group of people.
The bike/car options were a bit costly to implement, but we categorized it as a
very beneficial feature and we did not feel we could leave it out.

\section{Features not implemented}
\label{UIA-NI}
% Remove this section if it is empty of content anyway.
This section presents some of the features we have chosen not to implement.
These features are not in the final program, because we did not feel there were
compelling arguments for implementing them.

Features that we wanted to implement, but did not make it into the final version,
will be discussed in chapter \class{\nameref{PRC}} on page \pageref{PRC}.

\subsection{Choice of roads to be displayed}
\label{UIA-NI-CRD}
We chose not to implement the option of selecting which roads to be displayed.
In a sense our program already does this by showing more detail the further
zoomed in the map is. It could become very confusing if the user had the option
of selecting roads, because the graphical user interface could become very
cluttered, if all the roads were listed.

One advantage to this could be the option for the user to select which roads
should be included in the route planning - ferries, highways, bridges etc. 

\subsection{Smooth scrolling}
\label{UIA-NI-SS}
We made an attempt to let the keyboard arrows scroll smoothly over the map, but
we could not implement it to work fast enough, so the user would experience
lockups and the user interface hanging at some points. A solution to this would
be to save map at images that you can scroll across - this would be faster, but
would require more disk space. Because we store the data the way we do, which 
forces us to draw every line individually every time the user moves the viewport, 
we cannot do this fast enough.

In the end, we decided the benefit of the smooth scrolling was not big enough
for us to spend a lot of time implementing this feature. The cost of changing that  
much way we draw the roads, was simply too high compared to the benefits.

\subsection{Dynamic route finding}
\label{UIA-NI-DRF}
In the final project description, a dynamic route finding feature was
suggested. If we had implemented the suggested dynamic route finding feature, a
user would be able to mark a spot and then whenever he moused over a node on the
map, he would find the route instantly.

We considered implementing this as we thought it was a
nice feature to have, but it conflicted with the algorithm we use for
calculating the route. More about the algorithms can be read in section
\ref{IMPL-DVA} \class{\nameref{IMPL-DVA}} on page \pageref{IMPL-DVA}.

\chapter{Implementation}
\label{IMPL}
\section{Dijkstra vs A-star}
\label{IMPL-DVA}

\section{Evaluator}
\label{IMPL-EVA}

\section{Quadtree}
\label{IMPL-QT}

\section{Serialization}
\label{IMPL-SERI}

\section{UTM-convertion}
\label{IMPL-UTM}

\section{Mousezoom}
\label{IMPL-MZ}

\section{Floats}
\label{IMPL-F}
\chapter{UML-diagrams}
\label{UML}

Figure \ref{control-flow} shows our implementation of the MVC architecture.
Because we have exactly one window, we chose to name our view and controller ``\class{View}'' and 
``\class{Control}'', respectively. The same goes for the models. We only have one data 
source (Krak's data-set), and because of that our model is called ``\class{Model}''. We could 
have named these three classes something different that may have been more meaningful 
in terms of the \class{Map of Denmark} context, but we chose these name in order to make 
our architecture clear.

We wanted to keep the models as ``skinny'' as possible, although our \class{Model} is quite long. 
But the amount of (public) methods is small, so looking at it from outside, it is a skinny model. 
The reason we wanted to keep the models skinny, is because it isn't the model's responsibility to 
deliver the same data in different ways, do a lot of calculations or stuff like that. It only acts like a 
``middle-man'', delivering data to other classes. If some class want the data in another way, they 
will need to convert it themselves.

\class{Control} makes sure the models and the view can understand each other. The view 
only knows about pixels, but it has no idea about UTM coordinates. The model only knows 
about UTM coordinates, but doesn't know anything about pixels. So for getting these two to 
communicate, we need to convert pixels to UTM and vice versa.

We created some helper-classes (\class{PointMethods} and \class{RectangleMethods}), 
which are located in the \class{utils} package. These take care of checking whether a point 
is within the maximal bounds of the map, converting a pixel coordinate to a UTM coordinate 
and vice versa. We did this for being able to do this in several files, without the need to have 
duplicate code. For our program right now, this isn't really a problem, as we only have one 
model, one controller and a view. But if we were to have more, we would either need to copy 
these helper-methods into the other classes (BAD), or put them in a separate class. But even 
though we only have one of each, the helper-classes are still an advantage, as
they make the code cleaner and easier to maintain and test, as we can test these helper-classes.

So in essence, we have two parts (the model and the view) that need to communicate 
somehow, in order to display the data on the screen. But they speak different languages, so 
we put in a middle-man (the controller), responsible for the communication between the two.

\section{Control flow}
\label{UML-CF}
The easiest way to understand our flow through the program and how the individual parts 
talk together, is by using an example. Let us say the user has already clicked the map and 
placed a pin. Now the user clicks on the map again to place another pin.

\begin{figure}[h!]
\centering
\includegraphics[width=0.75\linewidth]{images/SimpleControlFlow}
\caption{The basic control flow of our Map program}
\label{control-flow}
\end{figure}

The controller has placed listeners in the view, so when a MouseClicked event is
thrown, the controller is called. First it checks if there is another pin at the spot of the click. If 
there is, this will be removed, and the model is told to clear the route. If there is still over two 
pins placed, the model is asked to calculate a new route.

If there isn't a pin where the user clicks, we place a new pin. If the user has placed two or 
more pins, the controller calls its own findPath-method from point 1 to point 2, point 2 to 
point 3 and so on. The findPath()-method tells the model to find a path between the two 
points given.

The model then asks a helper-class to find a path, using the A* algorithm, and provides it 
with the graph and the two points. When a path is found, it is saved in the model, ready 
for use in the controller.

The final step is getting the view to draw the route. The controller gets ready for a repaint, 
by fetching the route from the model. Then it passes this route to the view's repaint-method, 
and the view paints the road.
\chapter{Tests}
\section{WB: closestEdge}

\section{JUnit}

\section{System test}
\chapter{Manual}
\label{MAN}

This manual will explain the basic use of the map, as well as its
advances features. The images are from mac osx and might look different on other
operating systems.

\section{Navigation}
\label{MAN-N}
At startup, the entire country of Denmark is shown. Now, we can move around the
map with both the diretion buttons in the top left corner, and the arrow keys.
To move west, click the button
\includegraphics[height=1.3em]{images/westbutton.png} or press the left arrow
key. This goes for all four directions.

\begin{figure}[h!]
\centering
\includegraphics[width=1\linewidth]{images/man-move.png}
\caption{Moving the map}
\label{MAN-Z-COP}
\end{figure}

\section{Zoom}
\label{MAN-Z}
To zoom-in on the map, click the
\includegraphics[height=1.5em]{images/zoominbutton.png} button in the navigation
panel.

To zoom-in on a specific area, click and drag a rectangle around that
area on the map. A blue transparent rectangle will show you what you have
selected. To zoom-in, release the mousebutton.

For example, if you want to zoom-in on Copenhagen,
click the upper left corner of the city, and drag the cursor to the lower right
corner.

\begin{figure}[h!]
\centering
\includegraphics[width=1\linewidth]{images/man-copenhagen.png}
\caption{Zooming in on Copenhagen}
\label{MAN-Z-COP}
\end{figure}

To zoom-out of the map, click the
\includegraphics[height=1.5em]{images/zoomoutbutton.png} button. To return to
the startup view (showing the entire map of Denmark) press the \class{ESC} button.

\section{Route find}
\label{MAN-RF}
To find a route from one point to another, you must specify a start and an end
location. Click anywhere on the map to choose your start location. A light blue
marker will appear, containing the number ``1''. To choose your end location,
click at another location. A marker containing the number ``2'' will appear. The
best route from 1 to 2 will be calculated, and shown on the map as a blue path.
To extend your route with more markers, you can click at a new location on the
map. You can repeat this an unlimited amount of times. You can delete one of
your markers by clicking at its root. To delete all marker, click the button
'clear markers' or press \class{c}.

\begin{figure}[h!]
\centering
\includegraphics[width=1\linewidth]{images/man-route.png}
\caption{A route from 1 to 2 has been calculated}
\label{MAN-RF-IMG}
\end{figure}

\section{Bike/car}
\label{MAN-BC}
When calculating routes, it is important to specific which form of
transportation you wish to use. Choose your preferred style of transportation by
selecting the corresponding radio button.

\begin{figure}[h!]
\centering
\includegraphics[height=2em]{images/radiobuttons.png}
\caption{The radiobuttons}
\label{MAN-BC-IMG}
\end{figure}

To form of transportation you use will have big insfluence on what route is
calculated. A route for a car will use highways, while a route for bicycles will
use paths.

\section{Resize}
\label{MAN-RS}
To resize the map, drag the window as you would with any other application. The
map will automatic adjust to the new size of the window.
\begin{figure}[h!]
\centering
\includegraphics[width=1\linewidth]{images/man-resize.png}
\caption{The map has been resized}
\label{man-radio}
\end{figure}
\chapter{Product conclusion}
\label{PRC}
We believe our product live up to the project requirements described in chapter
\ref{BG} \class{\nameref{BG}}. Our program also runs fast enough when using the
interface. We tested this on both a fast computer and an older, slower computer
and we did not feel it was cumbersome to use the program, even on the slower
computer. We did not thoroughly test this however, and we cannot guarantee that
there will not be circumstances in terms of RAM-usage where our program will run
slowly.
 
The system tests were done on a Windows 7 computer and because of this, we
cannot be certain that all features work on other operative systems. However, 
two of our group members use MacOSX and they have experienced no problems 
with our program.

We did all of our testing internally. For a even more thorough testing of our
software, we could have asked other groups to test the software for us, so that
we could get a different look on the tests. We did no usability testing
throughout the project, and while our user interface is simple, we should have
tested to make sure we designed our interface in the best way possible.

\section{Issues in the software}
\label{PRC-BUGS}
We have discovered a few bugs our software that did not show itself through our
testing, but instead manifested with regular use of the software.

We have a bug in our zoom-function, which enables the user to zoom very far in
by zooming in, then resizing the window. This caused the program to be locked at
a very far zoomed in view, but we chose to let the user zoom out when they are
zoomed far in. The problem with zooming too far in can still happen, but the
view is not locked when it happens.

\section{Future features}
\label{PRC-FF}
If we had had time, we would have included more features in our program. We feel
our program could still do with some fine tuning for it to be of real use
outside of living up to the requirements.

We would have liked to have implemented a ``road search'' function, enabling the
user to search for the name of a road and place a marker there. In addition, we
would have like to implement the option for the user to view a route description
with turns.

To make the map look nicer, we attempted to implement a function for the map to
create a ``border'' splitting up the land and sea, so that it was more obvious
what is land and what is water. We would also have liked to make a logo for our
program, but we did not get around to implementing it.

\chapter{Group norms}
\label{GN}
We wrote a constitution before we the project was handed out to us. In this
constitution we describe what we require of eachother and ourselves in terms of
working on the project. We felt we made some rather strict requirements so that
we were sure to get some work done. We later changed
them a bit once our schedule cleared up from lectures early May.

This is the requirement part of our constitution:
\begin{itemize}
  \item Check mail at least once a day
  \item Tell the rest of the group in time if you have trouble getting things done on
  time.
  \item Admit when you are not done on time.
  \item Do not waste time when we have meetings.
  \item Respect that different people work in different paces and different
  ways.
  \item We need to evaluate often.
\end{itemize}

We also tried to get a mean of our level of ambition. Our goal was always to do
what we could manage to do in the time frame that we had and without wasting
time. In other words, we strived for being done early, by planning what we really 
wanted to have, and if we had time left, it was a bonus.

\section{Meetings}
\label{GN-M}
We structured our work together in ``meetings''. A meeting was whenever we were
together working on the project - these were to be done at the ITU. The
structure of a meeting was simple:
\begin{itemize}
  \item Leader of the meeting presents his plans, if any, he has for today. 
  \item Leader selects someone to write down what happened at the meeting.
  \item What have we done since last time?
  \item Who does what today?
  \item Work today.
  \item Fifteen minutes before work ends: decide on homework for next time and
  select leader of the next meeting.
\end{itemize}
Before our lectures ended in early May, we had meets Tuesdays and Fridays from
10 AM to 4 PM. After lectures ended, we felt only meeting Tuesdays and Fridays
would be too little time spent in meetings. So we decided to make our meetings
one and a half hours shorter, but instead meet on Mondays, Wednesdays and
Fridays from 10 AM to 2:30 PM.
\chapter{Diary}
3-4
\chapter{Worksheets}
For this project we made a meeting document for all our meetings. We used
these to make sure we kept to agreements, homework, had structured meetings. It
was important for us to have these, because we wanted to have something to fall
back on if we had disagreements and for us to have an easy way of being on top
with everyone's homework. They are all written in Danish, becuase our meetings
happened in Danish.

The documents are all structured in the following way: 
\begin{itemize}
  \item Leader of meeting
  \item Keeper of minutes
  \item Homework done before today
  \item Work to be done during the meeting
  \item Summary of the meeting
  \item Homework
\end{itemize}

All of our meeting documents can be found in appendix \ref{APP-WS-MD} on page
\pageref{APP-WS-MD}.

\chapter{Process description and reflection}
\chapter{ProcessReflection}
% I feel I have read this before?
% This needs more paragraphs. It is a mess.
Our teachers had a few requirements for features in the project. The
requirements were presented to us during development, in 3 steps. This made it
easier for us to focus on getting the basic features of the program to work, but
it made it harder for us to plan ahead as well. At the very beginning of the
process, we started working on the class-design. We used quite a lot of time on
the class design, and discussed for hours every detail of the classes and their
methods. We did not stop working on the design until everyone agreed that the
design of the classes were optimal. This is perhaps one of the things that
worked best for us in the process. During the development of the program, we
have been very happy with this design, and it has made it easier for us work on
the code. While working on the implementation, we have prioritized  delegating
the different tasks as much as possible. A lot of the work has been done at
home, so it has been important for us each to work with our own specific task.
In the meetings, we discussed the different solutions, problems and other things
of interest. In the middle of part 2 of the project, the work in our group
started getting quite a few problems. We had a bit of a crisis when one of the
group�s members, Filip, got stressed and had to leave the project. Jens came to
the meetings too late, and found the meetings to be inefficient. We decided to
have an evaluation of the group work, where we discussed these problems. The
evaluation had a positive effect on the group work. We have learned, that it is
important to address these problems when they start affecting the work process.
We have worked concentrated on the final report in the latter part of the
project. A week before the final hand in we stopped working on the code
completely (code-freeze). This made it easier to work on the final report, and to hold
focus on the most important tasks. In this final week, we have had a detailed
schedule to make sure that we did not fall behind.

\label{PDF}
\begin{thebibliography}{99}
\bibitem{algs4}
Robert Sedgewick and Kevin Wayne. \emph {Algorithms, Fourth Edition}. Preliminary Edition Fall 2010. Addision-Wesley 2010.
\end{thebibliography}
\pagestyle{plain}
\appendix
\appendixpage
\addappheadtotoc
\chapter{Tests}
This section covers tables and other appendices for our \class{Tests} chapter in
the report.
\section{PointMethods test table}
\label{APP-TE-PM}
\begin{centering}
\begin{longtable}{|p{3.5cm}|p{3cm}|p{3cm}|p{3cm}|}
\hline
Test description & Example & Expected Output & Actual Output\\
\hline
\hline
testPixelToUTM & view:(0, 0, 800, 600) model:(0, 0, 4000, 3000) point(80, 350) &
(400, 1250) & (400, 1250)\\
\hline
\hline
testUTMToPixel & view:(0, 0, 800, 600) model:(0, 0, 4000, 3000)
point:(pixelToUTM(80, 350)) & (80, 350) & (80, 350)\\
\hline
\hline
testPointOutOfBounds & view:(0, 0, 800, 600) & & \\
\hline
Inside & Point a:(100, 400) & a = (100, 400) & (100, 400)\\
\hline
Far left & Point a:(-100, 400) & a = (0, 400) & a = (0, 400)\\
\hline
Far right & Point a:(1200, 400) & a = (800, 400) & a = (800, 400)\\
\hline
Far up & Point a:(100, -300) & a = (100, 0) & a = (100, 0)\\
\hline
Far down & Point a:(100, 2000) & a = (100, 600) & a = (100, 600)\\
\hline
Far left \& far south & Point a:(-100, -200) & a = (0, 0) & a = (0, 0)\\
\hline
\end{longtable}
\end{centering}

\section{RectangleMethods test table}
\label{APP-TE-RM}

The table in this section shows the full extent of our JUnit test coverage of
the \class{RectangleMethods} class. Any variable that is described in the
example column in the row of the method name is used in all the tests for that
method.

\input{chapters/appendix/tables/RectangleMethods}
\chapter{Worksheets}
\label{APP-WS}
\section{Spreadsheet diary}
\label{APP-SS}
\begin{centering}
\begin{longtable}{|p{0.1\textwidth}|p{0.1\textwidth}|p{0.35\textwidth}|p{0.35\textwidth}|}
\hline
Dato & Medlem & L�st opgave & Personlig note\\
\hline
\hline
3/8/ 2011 & Jakob & Oprettede dokument & \\
\hline
3/14/ 2011 & Niklas & Lavede .gitignore, refaktorede kode(+prettify) & \\
\hline
3/15/ 2011 & Emil & Implementeret QuadTree & Der er noget galt i Map, da op/ned
scroll virker omvendt, desuden en mystisk implementation i Direction Enum'et, det skal �ndres til en switch direkte i Map.\\
\hline
3/17/ 2011 & Jakob & Fixede merge problems i control & Forst�r ikke helt det med
file. Skal have det forklaret p� m�det. Skal bede Niklas om ikke at lave min hjemmeopgave for mig! :D\\
\hline
3/22/ 2011 & Jakob & Lavede noget p� mousezoom & \\
\hline
3/24/ 2011 & Emil & Vejnavn ved klik & Er stadig nogle bugs der skal fixes\\
\hline
3/25/ 2011 & Emil & Resize og thickness & \\
\hline
3/27/ 2011 & Emil & Vejnavn... & Der st�r nu ingen tekst nederst n�r man har
musen over 200m fra en vej, desuden ser firkanten en del federe ud (gennemsigtig bl�) n�r man er ved at zoome\\
\hline
3/28/ 2011 & Jakob & Testede p� hjemmecomputer & Begyndte at kigge p� LateX.
Mulighed for at zoome ud til start ville v�re bonus mega awesome!\\
\hline
3/28/ 2011 & Niklas & Optimerede RAM forbrug & Double -\textgreater Float\\
\hline
3/31/ 2011 & Jakob & Tegnede hvordan vi laver pixels om til UTM og skrev det
meste af det jeg skulle skrive & Vi har nem mulighed for at lave god formatering af vores tekst - skal huske at vise hvordan fredag. Ville v�re fantastisk, hvis vi kunne f� lagt tingene ind i ordenlige packages - so much shit in default package! Diskuter PixelToUTM e.x/width*width \\
\hline
3/31/ 2011 & Filip & Skrev afsnit om Line og tilf�jede tykkelser p� veje i map &
\\
\hline
4/5/ 2011 & Emil & refactoring & Flyttet noget logik fra QuadTree til Model\\
\hline
4/7/ 2011 & Jakob & Refactorede control og oprettede nye klasser - lavede JUnit
& \\
\hline
4/12/ 2011 & Niklas & Tilf�jede multithreading loading af serialiserede filer. &
Langsomme computere kan lave foresp�rgelserne i nogle af quadtrees f�r de er loadet. UPDATE d. 15/4/2011 - det er fixet, ved at returnere de edges og nodes som er fundet i de fundne quadtrees.\\
\hline
4/23/ 2011 & Jakob & Lavede clearpins & \\
\hline
4/28/ 2011 & Niklas & Lavede en default evaluator & Hidtil var der ikke nogen
bestemmelse om hvad der skulle ske, hvis der ikke var valgt en evaluator. Jeg har nu lavet en Evaluator.DEFAULT, der bare refererer til Evaluator.CAR. Hvad skal der ske hvis brugeren skifter evaluator, efter ruten er lagt?\\
\hline
4/30/ 2011 & Niklas & Fixede serialisering & Det hjalp at serialisere til samme
fil, med samme stream. P� den m�de opst�r der ikke dublikater. Desuden er fylder
dataen mindre, det g�r hurtigere, og bruger v�sentligt mindre RAM (550MiB
=\textgreater 260MiB)\\
\hline
5/1/ 2011 & Niklas & For�rs-reng�ring af kode & Har flyttet klasser rundt til de
rigtige pakker. Har desuden gennemg�et koden sammen med FindBugs, og ryddet op i koden.\\
\hline
5/5/ 2011 & Niklas & Tilf�jede anti-aliasing & Problemet var at vi tilf�jede
anti-aliasing p� baggrunden ogs�, og derfor k�rte det langsomt. Pr�vede kun at bruge det p� stregerne, og det fungerer ganske udem�rket, og er stadig hurtigt.\\
\hline
5/6/ 2011 & Emil & Sk�nhed & Implementerede "tykke" veje med outline og et mere
differencieret quadtree PS: jeg har glemt at skrive herinde en masse gange\\
\hline
5/8/ 2011 & Jakob & LaTeX & Arbejdede med LaTeX og med smooth scrolling\\
\hline
5/14/ 2011 & Emil & White Box Test & Har skrevet white box tests til
getClosestNode metoden, mangler dog at lave det til JUnit da jeg skal vide hvordan jeg loader test-grafen, de andre tests der bruger den virker ikke ved mig\\
\hline
\end{longtable}
\end{centering}
\end{document}