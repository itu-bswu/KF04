\chapter{Group norms}
\label{GN}
We wrote a constitution before we the project was handed out to us. In this
constitution we describe what we require of eachother and ourselves in terms of
working on the project. We felt we made some rather strict requirements so that
we were sure to get some work done. This backfired a little bit, because we did
not keep all the agreements we made, but most of the worked and we later changed
them a bit once our schedule cleared up from lectures early May.

This is the requirement part of our constitution:
\begin{itemize}
  \item Check mail at least once a day
  \item Tell the rest of the group in time if you have trouble getting done on
  time.
  \item Admit when you are not done on time.
  \item Do not waste time when we have meets.
  \item Respect that different people work in different paces and different
  ways.
  \item We need to evaluate often.
\end{itemize}

We also tried to get a mean of our level of ambition. Our goal was always to do
what we could manage to do in the time frame that we had and without wasting
time. 
\section{Meetings}
\label{GN-M}
We structured our work together in ``meetings''. A meeting was whenever we were
together working on the project - these were to be done at the ITU. The
structure of a meeting was simple:
\begin{itemize}
  \item Leader of the meeting presents his plans, if any, he has for today. 
  \item Leader selects someone to write down what happened at the meeting.
  \item What have we done since last time?
  \item Who does what today?
  \item Work today.
  \item Fifteen minutes before work ends: decide on homework for next time and
  select leader of the next meeting.
\end{itemize}
Before our lectures ended in early May, we had meets Tuesdays and Fridays from
10 AM to 4 PM. After lectures ended, we felt only meeting Tuesdays and Fridays
would be too little time spent in meetings. So we decided to make our meetings
one and a half hours shorter, but instead meet on Mondays, Wednesdays and
Fridays from 10 AM to 2:30 PM.
