\chapter{Group norms}
% I feel I have read this before?
% This needs more paragraphs. It is a mess.
Our teachers had a few requirements for features in the project. The
requirements were presented to us during development, in 3 steps. This made it
easier for us to focus on getting the basic features of the program to work, but
it made it harder for us to plan ahead as well. At the very beginning of the
process, we started working on the class-design. We used quite a lot of time on
the class design, and discussed for hours every detail of the classes and their
methods. We did not stop working on the design until everyone agreed that the
design of the classes were optimal. This is perhaps one of the things that
worked best for us in the process. During the development of the program, we
have been very happy with this design, and it has made it easier for us work on
the code. While working on the implementation, we have prioritized  delegating
the different tasks as much as possible. A lot of the work has been done at
home, so it has been important for us each to work with our own specific task.
In the meetings, we discussed the different solutions, problems and other things
of interest. In the middle of part 2 of the project, the work in our group
started getting quite a few problems. We had a bit of a crisis when one of the
group's members, Filip, got stressed and had to leave the project. Jens came to
the meetings too late, and found the meeting to be inefficient. We decided to
have an evaluation of the group work, where we discussed these problems. The
evaluation had a positive effect on the group work. We have learned, that it is
important to address these problems when they start affecting the work process.
We have worked concentrated on the final report in the latter part of the
project. A week before the final XX we stopped working on the code completely
(code-freeze). This made it easier to work on the final report, and to hold
focus on the most important tasks. In this final week, we have had a detailed
schedule to make sure that we did not fall behind.
