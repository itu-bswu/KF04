\begin{centering}
\begin{longtable}{|p{0.1\textwidth}|p{0.1\textwidth}|p{0.35\textwidth}|p{0.35\textwidth}|}
\hline
Dato & Medlem & L�st opgave & Personlig note\\
\hline
\hline
3/8/ 2011 & Jakob & Oprettede dokument & \\
\hline
3/14/ 2011 & Niklas & Lavede .gitignore, refaktorede kode(+prettify) & \\
\hline
3/15/ 2011 & Emil & Implementeret QuadTree & Der er noget galt i Map, da op/ned
scroll virker omvendt, desuden en mystisk implementation i Direction Enum'et, det skal �ndres til en switch direkte i Map.\\
\hline
3/17/ 2011 & Jakob & Fixede merge problems i control & Forst�r ikke helt det med
file. Skal have det forklaret p� m�det. Skal bede Niklas om ikke at lave min hjemmeopgave for mig! :D\\
\hline
3/22/ 2011 & Jakob & Lavede noget p� mousezoom & \\
\hline
3/24/ 2011 & Emil & Vejnavn ved klik & Er stadig nogle bugs der skal fixes\\
\hline
3/25/ 2011 & Emil & Resize og thickness & \\
\hline
3/27/ 2011 & Emil & Vejnavn... & Der st�r nu ingen tekst nederst n�r man har
musen over 200m fra en vej, desuden ser firkanten en del federe ud (gennemsigtig bl�) n�r man er ved at zoome\\
\hline
3/28/ 2011 & Jakob & Testede p� hjemmecomputer & Begyndte at kigge p� LateX.
Mulighed for at zoome ud til start ville v�re bonus mega awesome!\\
\hline
3/28/ 2011 & Niklas & Optimerede RAM forbrug & Double -\textgreater Float\\
\hline
3/31/ 2011 & Jakob & Tegnede hvordan vi laver pixels om til UTM og skrev det
meste af det jeg skulle skrive & Vi har nem mulighed for at lave god formatering af vores tekst - skal huske at vise hvordan fredag. Ville v�re fantastisk, hvis vi kunne f� lagt tingene ind i ordenlige packages - so much shit in default package! Diskuter PixelToUTM e.x/width*width \\
\hline
3/31/ 2011 & Filip & Skrev afsnit om Line og tilf�jede tykkelser p� veje i map &
\\
\hline
4/5/ 2011 & Emil & refactoring & Flyttet noget logik fra QuadTree til Model\\
\hline
4/7/ 2011 & Jakob & Refactorede control og oprettede nye klasser - lavede JUnit
& \\
\hline
4/12/ 2011 & Niklas & Tilf�jede multithreading loading af serialiserede filer. &
Langsomme computere kan lave foresp�rgelserne i nogle af quadtrees f�r de er loadet. UPDATE d. 15/4/2011 - det er fixet, ved at returnere de edges og nodes som er fundet i de fundne quadtrees.\\
\hline
4/23/ 2011 & Jakob & Lavede clearpins & \\
\hline
4/28/ 2011 & Niklas & Lavede en default evaluator & Hidtil var der ikke nogen
bestemmelse om hvad der skulle ske, hvis der ikke var valgt en evaluator. Jeg har nu lavet en Evaluator.DEFAULT, der bare refererer til Evaluator.CAR. Hvad skal der ske hvis brugeren skifter evaluator, efter ruten er lagt?\\
\hline
4/30/ 2011 & Niklas & Fixede serialisering & Det hjalp at serialisere til samme
fil, med samme stream. P� den m�de opst�r der ikke dublikater. Desuden er fylder
dataen mindre, det g�r hurtigere, og bruger v�sentligt mindre RAM (550MiB
=\textgreater 260MiB)\\
\hline
5/1/ 2011 & Niklas & For�rs-reng�ring af kode & Har flyttet klasser rundt til de
rigtige pakker. Har desuden gennemg�et koden sammen med FindBugs, og ryddet op i koden.\\
\hline
5/5/ 2011 & Niklas & Tilf�jede anti-aliasing & Problemet var at vi tilf�jede
anti-aliasing p� baggrunden ogs�, og derfor k�rte det langsomt. Pr�vede kun at bruge det p� stregerne, og det fungerer ganske udem�rket, og er stadig hurtigt.\\
\hline
5/6/ 2011 & Emil & Sk�nhed & Implementerede "tykke" veje med outline og et mere
differencieret quadtree PS: jeg har glemt at skrive herinde en masse gange\\
\hline
5/8/ 2011 & Jakob & LaTeX & Arbejdede med LaTeX og med smooth scrolling\\
\hline
5/14/ 2011 & Emil & White Box Test & Har skrevet white box tests til
getClosestNode metoden, mangler dog at lave det til JUnit da jeg skal vide hvordan jeg loader test-grafen, de andre tests der bruger den virker ikke ved mig\\
\hline
\end{longtable}
\end{centering}