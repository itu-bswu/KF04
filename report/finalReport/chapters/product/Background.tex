\chapter{Background}
\label{BG}
\section{Problem area}
\label{BG-PR}
Over the last decade people have switched from traditional roadmaps 
to using the web-maps. This is a change without any negative side-effects. 
The online services remove all the problems with determining the quickest 
route between two points and you spend no time browsing the pages of 
the map to find what you need. With the popular smartphones the online 
map is even more useful, since you no longer need to prepare your trip 
before you leave.

The online maps have now been used for many years and haven`t been slow 
at adopting new features to improve their usability. They have both 
implemented satellite-maps that allow us to browse the entire planet 
from above, and lately the feature called Google Street View has upped 
the stakes when allowing us to look at any direction from a given point 
of a road. The two maps that we use the most are Google Maps and the 
Danish map called Krak. These maps both have the mentioned features 
but slight differences in the way the user navigates and searches 
for routes.

Because of the widespread knowledge of the online maps, the users 
have been accustomed to certain features and ways of using the map. 
It is very important that we, with a new map program, use this knowledge 
to our advantage and don`t try to reinvent the wheel. By using some 
of the commonly used controls in our map, a user will be able to quickly 
adapt to our program and use it efficiently. 

\section{Requirements for the map}
Our teachers had a few requirements for features in the project, that we had to
obey. The requirements were presented to us during development, in 3 steps. This made
it easier for us to focus on getting the basic features of the program to work,
but it made it harder for us to plan ahead as well.  Here is a list of the
features we had to implement: 1) We had to make a visual representation of all
the roads from the dataset. 2) We had to draw different roads in different
colours. 3) We had to adjust our drawing of the roads to the windows size of
our GUI. 4) We had to make mouse zooming possible.  The user should be able to
drag the mouse from one corner of the map to another, in order to zoom in on
the selected area. 5) We had to implement a method to find a path from one road
in the map to another, and we had to allow the user to find that path by
clicking at each of the roads. We had to consider these requirements while
deciding how the program should be designed. Of course, some of the
requirements were so self explanatory that we would have made them regardless
of them being required. Only requirement 4 were against the design idea we
would have chosen ourselves. Instead of implementing the mouse zoom by dragging
a specific area, we would have made the user zoom in on the map using the
scrolling wheel. This would have been a more intuitive implementation, where
the current system is more precise.

\label{BG-R}
\section{Our requirements}
In the process of designing the program we made some requirements that we
wanted to make sure was met before making more advanced features. Since it was
required to make the user able to zoom-in on the map, we found it logical to
allow him to zoom-out, and scroll the map as well. With these basic features
covered, we decided which advanced featured we wanted implement. In order to
give the user a chance to find more specific places, we decided to show the
user the name of the road closest to the cursor. This means that the user can
get orientated without clicking or pressing any button. The most interesting
feature of our program perhaps is the ability to choose whether you want to
travel by car or by bicycle. Many Danish people uses bicycles to travel,
therefore this feature is very relevant. Because this is a feature that
wouldn�t make sense on a world map, it makes our program superior to its
international competitors. We also decided to let the user create routes with
an unlimited amount of waypoints. This makes our map well suited for planning
bicycling trips where you want to reach more than one destination. Because of
this feature, we required our path finding algorithm to be fast enough to make
this possible. We wanted to make sure that the shape of Denmark is recognizable
when the user has zoomed out to show the entire country. There is a balance
between showing a big amount of roads and the elay when navigating. We wanted
to make the user able to navigate the map with a reasonable amount of delay.
The fact that we chose specific requirements for our program gave us two big
advantages. It both made the working process easier, and it made our choices of
features more consistent.  Both because we knew which direction we wanted to go
with program. In the process of creating the program, we changed and created
new requirements to ourselves. In the last part of the coding process, we
decided on our final requirements and worked towards completing them.
\subsection{Project requirements}
\label{BG-R-PR}
\subsection{Our own requirements}
\label{BG-R-OOR}

\section{Data set}
\label{BG-DS}
We have been provided with a dataset of roads and intersections in Denmark 
from Krak. Additionally we got some code for loading the data in from the 
text files. We have only made minor changes to the code for loading the data.
\subsection{UTM-coordinates}
\label{BG-DS-UTM}
It is important to note that the KrakNodes are in UTM-32 coordinates. When 
using the UTM standard the origo is placed at the south-west corner. These 
coordinates need some conversion when using in Java since the origo is placed 
differently.
\subsection{Graph}
\label{BG-DS-G}
When the data has been loaded it is stored as a Graph containing KrakNodes 
and KrakEdges. The KrakEdges are the road segments and contains the name 
of the road, an estimated drive time, a direction of traffic and references 
to the two KrakNodes that are at either end of the road.  The KrakNode itself 
contains only the coordinates for the point.
The Graph itself contains a number of useful methods for searching the data 
like getting all edges that is connected to a KrakNode. We will be using these 
methods extensively throughout the project both for drawing the map and for 
finding the route between two points.
\section{MVC structure}
\label{BG-MVC}
