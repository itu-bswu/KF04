\chapter{Background}
\label{BG}
\section{Problem area}
\label{BG-PR}
Over the last decade people have switched from traditional roadmaps 
to using the web-maps. This is a change without any negative side-effects. 
The online services remove all the problems with determining the quickest 
route between two points and you spend no time browsing the pages of 
the map to find what you need. With the popular smartphones the online 
map is even more useful, since you no longer need to prepare your trip 
before you leave.

The online maps have now been used for many years and haven`t been slow 
at adopting new features to improve their usability. They have both 
implemented satellite-maps that allow us to browse the entire planet 
from above, and lately the feature called Google Street View has upped 
the stakes when allowing us to look at any direction from a given point 
of a road. The two maps that we use the most are Google Maps and the 
Danish map called Krak. These maps both have the mentioned features 
but slight differences in the way the user navigates and searches 
for routes.

Because of the widespread knowledge of the online maps, the users 
have been accustomed to certain features and ways of using the map. 
It is very important that we, with a new map program, use this knowledge 
to our advantage and don`t try to reinvent the wheel. By using some 
of the commonly used controls in our map, a user will be able to quickly 
adapt to our program and use it efficiently. 

\section{Requirements for the map}
\label{BG-R}
\section{Our requirements}
\subsection{Project requirements}
\label{BG-R-PR}
\subsection{Our own requirements}
\label{BG-R-OOR}

\section{Data set}
\label{BG-DS}
We have been provided with a dataset of roads and intersections in Denmark 
from Krak. Additionally we got some code for loading the data in from the 
text files. We have only made minor changes to the code for loading the data.
\subsection{UTM-coordinates}
\label{BG-DS-UTM}
It is important to note that the KrakNodes are in UTM-32 coordinates. When 
using the UTM standard the origo is placed at the south-west corner. These 
coordinates need some conversion when using in Java since the origo is placed 
differently.
\subsection{Graph}
\label{BG-DS-G}
When the data has been loaded it is stored as a Graph containing KrakNodes 
and KrakEdges. The KrakEdges are the road segments and contains the name 
of the road, an estimated drive time, a direction of traffic and references 
to the two KrakNodes that are at either end of the road.  The KrakNode itself 
contains only the coordinates for the point.
The Graph itself contains a number of useful methods for searching the data 
like getting all edges that is connected to a KrakNode. We will be using these 
methods extensively throughout the project both for drawing the map and for 
finding the route between two points.
\section{MVC structure}
\label{BG-MVC}
