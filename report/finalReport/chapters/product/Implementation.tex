\chapter{Implementation}
\label{IMPL}
\section{Dijkstra vs A-star}
\label{IMPL-DVA}
In preparations to implementing the path finding feature to our program, we knew 
two possible choices. They are named Dijkstra and A* (A-star), and are quite 
similar but behave a little different.

The Dijkstra algorithm uses a minimum priority queue to find the shortest path 
from a given node to every other node by looking at the edges connecting nodes. 
The program will take a node from the priority queue and add all the other nodes 
that are connected from the current to the priority queue. The priority queue 
takes a value to the node and this should be the distance to the current node 
plus the length of the edge between the two. Since the priority is made to return 
the node with the smallest value associated with it, the next node in line will 
always be the one which is closest to the start node. This procedure continues 
until all nodes have been visited and by logging what edge led to all the nodes 
it is possible to trace back the route to the start node.

This algorithm is great if need to find the distance from one point to many other 
points, but can be quite slow since it just searches in all directions without 
concerns to the direction of the target node.

This is where A* comes in handy. The A* algorithm is a modification to Dijkstra 
that also looks at the estimated distance from the given node when determining 
the value for the priority queue. When using the geographical distance as a 
measure of �best route� the value would be the current distance from the start 
node plus the direct distance to the target (as if there were a road directly to 
the target). With this subtle change the algorithm will prioritize nodes that are 
relatively closer to the target than those that are in the other direction. This 
makes the algorithm much faster since it will not pay much attention to the roads 
that are not in the direction of the target.
We have decided to use the A* algorithm since we only calculate routes between 
two distinct nodes and therefore don`t need the route from the start node to all 
others. The time reduction that A* gives is also a definite plus since no user 
wants to sit and wait too long for the program to find the route.

\section{Evaluator}
\label{IMPL-EVA}
In order to make our path finding algorithm flexible enough for different 
interpretations of the �best route�, we have added an entity called \class{Evaluator}. 
This is an object that has the responsibility of evaluating a node relative to 
the target node. The \class{Evaluator} also has the responsibility of calculating the 
heuristics that the A* algorithm relies on. By using the \class{Evaluator} we are able to 
use the same path finding algorithm for two very different tasks, namely the 
biking route and the car route. The major difference between these is that the 
bike uses the distance and the car uses the total drive time. This implementation 
is also a good example of making our code ready for future features, since if we 
needed to add other means of transportation or simply variations of the ones we 
have, we would only need to create new \class{Evaluator} objects and not change a single 
line of code in the A* algorithm.

\section{Quadtree}
\label{IMPL-QT}

\section{Serialization}
\label{IMPL-SERI}

\section{UTM-convertion}
\label{IMPL-UTM}

\section{Mousezoom}
\label{IMPL-MZ}

\section{Floats}
\label{IMPL-F}