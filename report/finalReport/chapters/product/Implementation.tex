\chapter{Implementation}
\label{IMPL}
Our implementation uses the MVC structure to create a practical class design. We
have prioritized making our design as simple and intuitive as possible. We used
object oriented principles to assure that our design has both low connection
between classes (loose coupling) and a high degree of cohesion.

By following these principles, we have achieved implementation with little code
duplication and with good possibility of extension.

Some of the basic thoughts on our design is explained in figure
\ref{fig:BasicDesign} and figure \ref{fig:PathfindingStructure}.

\begin{figure}[h!]
\centering
\includegraphics[width=1\linewidth]{images/BasicDesign}
\caption{The basic design of our implementation. The controller gets a
collection of lines and the view draws them.}
\label{fig:BasicDesign}
\end{figure}

\begin{figure}[h!]
\centering
\includegraphics[width=0.5\linewidth]{images/PathfindingStructure}
\caption{The pathfinding part of our implementation. The static A* function
takes an Evaluator, a Graph, a start \class{note} and an end \class{note} and
returns a path.}
\label{fig:PathfindingStructure}
\end{figure}

This rest of this chapter describes how we have implemented some of the more
interesting features of the software. We aim to describe it to enough detail that this
chapter can serve as a guideline for implementing the functionalities we
describe.

\section{UTM-conversion}
\label{IMPL-UTM}
When the graphical user interface part of the map tries to communicate with the
model through control, some conversions of the different kind of values are
necessary. Both when going from coordinates in the Java-coordinate system to
UTM32-coordinates and back.

We need to convert the values when we want to use the mousezoom and when we want
to place the markers for pathfinding. We get an input on the graphical user
interface when we use mouse zoom and this needs to be converted to
UTM-coordinates so we can create the new boundaries of the zoomed rectangle.

When we place markers for pathfinding, we do the same as when we mouse zoom,
but instead we store the point as UTM32-coordinates and whenever we move the
map, we convert it back to pixel-coordinates so that we know where to draw the
markers.

The Java-coordinates have origo placed in the top left corner with the
y-coordinate increasing the further down the y-axis you go. UTM32-coordinates are a bit
different. UTM32 has origo in the bottom left corner and the y-coordinate
increasing the further up you go on the y-axis.

We have a utility class with methods for converting the points back and forth.
Class \class{PointMethods} uses the below calculations to convert the points.

Figure \ref{fig:UTMconversion} is an illustration of the conversion from pixel to UTM.

\begin{figure}[!ht]
\centering
\includegraphics[width=1\linewidth]{images/UTMillu}
\caption{Illustration of UTM conversion}
\label{fig:UTMconversion}
\end{figure}

We click on a point, 'a', on the canvas. To calculate the UTM coordinate
corresponding to that point, we use the formulas:

\begin{center}
$
UTM_x = bounds_x + \cfrac{a_x}{canvas_{width}} \times bounds_{width}
$
\end{center}

\begin{center}
$
UTM_y = bounds_y + \cfrac{canvas_{height} - a_y}{canvas_{height}} \times
bounds_{height}
$
\end{center}

To convert from UTM to pixels, we use the same formulas but reversed:

\begin{center} 
$
Pixel_x = \cfrac{(a_x - bounds_x)}{bounds_{width}} \times
canvas_{width}
$
\end{center}

\begin{center}
$
Pixel_y = canvas_{height} - \cfrac{a_y - bounds_y}{bounds_{height}} \times
canvas_{height}
$
\end{center}

\section{Mousezoom}
\label{IMPL-MZ}
We have implemented mouse zoom by using \class{mouseEvent}s on
our canvas. When the user presses the left mouse button, it generates a
\class{mousePressed} event. We record the position of the mouse at the time of
the \class{mousePressed} event and wait for the \class{mouseReleased} event.
When the left mouse button is released, it generates a \class{mouseReleased}
event. We use the positions of the mouse at the \class{mousePressed} event and
the \class{mouseReleased} event to calculate new bounds for the model.

Often the user will not drag a square that is in perfect ratio with the
canvas. If it does not have the same ratio as the canvas, we change the ratio
of the dragged square behind the scenes. We do this by either adding length or
width to the dragged square. We always make sure to at least show what was
inside the box the user dragged.

If the ratio of the dragged square is smaller than the canvas ratio, we make the
dragged box wider. If the ratio of the dragged square is larger, we make the
dragged box taller.

\section{Dijkstra vs A*}
\label{IMPL-DVA}
When planning the pathfinding feature, we had to decide between two algorithms:
Dijkstras algorithm\footnote{We had a lecture about this in the Algorithms and
Data Structures course and read about it in \cite[p.~556]{algs4}.} and
A*\footnote{The advisors explained how this was faster and how we should
change Dijkstra to turn it into A*.}. The latter is based on Dijkstras
algorithm, but achieves better performance in some situations by using
heuristics.

Dijkstras algorithm uses a minimum priority queue\footnote{The minimum priority
queue has been borrowed from \cite{httpalgs4}} to find the shortest path from a
given node to every other node by looking at the edges connecting the nodes. The
algorithm will take a node from the priority queue and add all the other nodes
that are connected to the current node to the priority queue. The priority queue
stores a value with the node, which is the distance to the current node plus the
length of the edge between the two. Since the priority is made to return the
node with the smallest value associated with it, the next node in line will
always be the one which is closest to the start node. This procedure continues
until all nodes have been visited and by logging what edge led to all the nodes,
it is possible to trace back the route to the start node.

This algorithm is great if we need to find the distance from one point to many other
points, but can be quite slow when using it for finding a path between two
nodes, since it just searches in all directions without considering which
direction the target node is.

This is where A* is different. The A* algorithm is a modification of Dijkstras
algorithm that also looks at the estimated distance from the given node when
determining the value for the priority queue. When using the geographical distance
as a measure of best route, the value would be the current distance from the start
node plus the direct distance to the target (as if there were a road directly to
the target). See figure \ref{ad} for an example. 

With this subtle change, the algorithm will prioritize nodes that are
relatively closer to the target rather than those that are in the other
direction. This makes the algorithm much faster since it will not pay much attention
to the roads that are not in the direction of the target.
We have decided to use the A* algorithm since we only calculate routes between
two distinct nodes and therefore don`t need the route from the start node to all
others. The time reduction that A* gives is also a definite plus since no user
wants to sit and wait too long for the program to find a route.

\begin{figure}[!ht]
\centering
\includegraphics[width=1\linewidth]{images/AstarVSDijkstra.png}
\caption{A* vs. Dijkstra example. The dashed lines are the direct distances,
used by A*. The Dijkstra algoritm will check the blue node before the red
node, because B+A \textless C+D. The A* algorithm will check the red node
before the blue node, because C+D+E \textless B+A+F.}
\label{ad}
\end{figure}

The smaller the difference from the direct distance is to the actual fastest
possible route, the more we will benefit from the A* algorithm. This
difference will often be bigger for the routes calculated for the car. This is
because the fastest route would be a straight highway, which often is far from
possible. We can therefore conclude that the A* algorithm typically is more
of a benefit when calculating routes for bicycling because they rely on the
distance.

\section{Evaluator}
\label{IMPL-EVA}
In order to make our path finding algorithm flexible enough for different 
interpretations of the ``best route'', we have added an class called
\class{Evaluator}. This is an object that has the responsibility of evaluating a node relative to 
the target node. 

The \class{Evaluator} also has the responsibility of calculating the 
heuristics that the A* algorithm relies on. By using the \class{Evaluator} we are able to 
use the same path finding algorithm for two very different tasks; the biking route and 
the car route. The major difference between these is that the bike's heuristics is based 
on the distance, whereas the car's heuristics is based on the total drive time. 

\begin{center}
$
dist(p_{1},p_{2})=\sqrt{(x_{1}-x_{2})^{2}+(y_{1}-y_{2})^{2}}
$
\end{center}

\begin{center}
$
time_{drive}=\frac{dist(p_{1},p_{2})}{1000\cdot \frac {speed_{max}}{60}}
$
\end{center}

We use these formulas to calculate the heurestics for bike routes and car
routes, respectively. There are currently three different evaluators in the
program, saved as static variables. These are called BIKE, CAR and HAS\_NAME.
The HAS\_NAME evaluator is used when getting the name of the road closest to the
mouse. There is also a fourth variable called DEFAULT that is a help to any
future developers if they are unsure of what to use. This variable is currently
just referring to CAR.

The Evaluator implementation is a good example of making our code ready for
future features, since if we needed to add other means of transportation or
simply variations of the ones we have, we would only need to create new
\class{Evaluator}s and not change a single line of code in the A* algorithm.

\section{Quadtree}
\label{IMPL-QT}
In order to improve the performance drawing of our map, we have implemented a
data structure called ``quadtree''. The quadtree divides our map into four
rectangles. In these four rectangles it divides our map again into smaller
rectangles, and we keep doing this, until we reach an amount of roads that is
manageable. When we want to retrieve data from the map, we can give the quadtree
a rectangle, and it will return all the roads within our smaller rectangles that
is contained in or intersects the given rectangle. This technique optimizes the
drawing of the roads, because we greatly limit the amount of roads we draw.

However, when viewing the entire map, this implementation does not help
us. Therefore, it is necessary to only to draw the bigger roads when zoomed out.
We have discussed two different ways of doing this.

The simplest solution would be to iterate over the roads returned and then
remove the roads we do not want to draw. This is illustrated by figure
\ref{IMPL-USQ}. However, it would be more efficient to sort the roads before we
put them in the quadtree, and then have them sorted in the quadtree. By doing
this, we will not have to iterate over all the roads and remove them all the
time.

\begin{figure}[h!]
\centering
\includegraphics[width=1\linewidth]{images/UnsortedQuadtree.png}
\caption{Unsorted Quadtree. A quadtree where the roads are not sorted}
\label{IMPL-USQ}
\end{figure}

We have discussed two different ways of sorting the roads in the quadtree. 

The first way relied on dividing the types of roads into different
quadtrees. With this implementation, we have one quadtree for each level of
detail we want to have and only search in the quadtrees that are needed for a
given rectangle. This is multiple quadtrees. See figure \ref{IML-MTQ}.

\begin{figure}[h!]
\centering
\includegraphics[width=1\linewidth]{images/MultiQuadtree.png}
\caption{Multi-quadtree. First node divides the roads into more
quadtrees.}
\label{IMPL-MTQ}
\end{figure}

The second way relied on putting the bigger roads at the top of the
quadtree when building it. Then we could specify at which depth we wanted to
search the quadtree. We called this 'the depth-controlled quadtree'. Illustrated
by figure \ref{IMPL-DCQ}.

\begin{figure}[h!]
\centering
\includegraphics[width=1\linewidth]{images/DepthcontrolledQuadtree.png}
\caption{The depthcontrolled quadtree.}
\label{IMPL-DCQ}
\end{figure}

The two methods both have advantages and disadvantages.

The depth-controlled quadtree would return more roads than the multiquadtree,
because the large roads (highways, etc.) would be in the outer rectangles, and
therefore be drawn more often.

The depth-controlled quadtree requires less RAM than the multiquadtree,
because it requires less instances of object \class{QuadTreeNode}.

The multiquadtree is slighty less efficient to search through than the
depth-controlled quadtree, because we must run through each quadtree
individually.

Looking at the advantages and disadvantages of these quadtrees, we have chosen
to implement the multiquadtree. We have done this because it makes it a lot
easier for us to layer the roads by their roadtype. Also it is the drawing of
the roads that slows down our implementation. The time needed for searching
through the quadtree is decided by the amount of roads in it, which affects the
height of the tree. When using more quadtrees, the amount of roads are smaller,
and because of that the height is smaller. Even when using multiple quadtrees,
the efficiency of the map is primarily affected by the number of roads drawn,
not the amount of quadtrees.

\section{Serialization}
\label{IMPL-SERI}
We observed that the user had to wait quite a long time for the program to
start. This was because every time we start the program, we loop through the
entire provided data set. This data set is huge, and because of this it takes
quite a lot of time to start the program. Because we need to load all this data,
the user is presented with a blank screen for a long time, before all this data
are loaded and the program starts.

We started looking for a way to speed up the loading process, so the user has a
map in front of him or her quickly, when the user starts the program. What takes
the most time is looping through the data and creating the needed datastructures
(quadtrees, the graph and so on), so if we could skip these steps or speed them
up, we could save a lot of time.

This is where serialization comes into play. By serializing an object, you
transform your object into something that can be passed around through streams.
So by serializing objects, you can save them to files. If the object to be
serialized contains references to other objects, these will also be serialized
(if they are Serializable / implements \class{java.io.Serializable}).

By doing this, we only need to build our datastructures the first time you start
the program. After the objects have been created, they are been serialized and
saved to files. The next time the user starts the program, we check whether the
data has been changed. We check this by checking the MD5 checksum of the file,
with \class{MD5Checksum}\footnote{We borrowed this code from \cite{MD5}}. When
we serialize the objects, we also save the checksum to a config file. If the
data has not been changed (i.e. the checksum is the same), we load the objects
that we saved, instead of making them all over. 

We serialize all the quadtrees, the graph and the maxbounds-object (specifying
the bounds of Denmark), as these are the objects we need for the program to run.
When saved to the harddrive, it is around 65MiB, which should not be an issue,
given the size of harddrives today. 

We save all these objects to one single file in order to keep the references. If
we did not do this, the references will be ruined, which can break the program.
We experienced problems with this, as \class{nodes} and \class{edges} are both
stored in the quadtrees and the graph. If we serialized and saved the quadtrees
in one file, and the graph in other file, a given \class{node} would be saved
twice, and when we load the data from the serialized files, the \class{node}
will exist twice, and it won't be the same object. But if we save the objects to
a file through the same stream, we only get one of each, which leads to less RAM
usage, less harddrive usage, faster loadtimes and hopefully fewer bugs.

\subsection{Threading}
By serializing our main objects, we cut several seconds of our load times. But
we can do it even faster. We are serializing several objects, but only few of
them are needed right from the start of the program. So what we can do to speed
it up even more, is loading the few necessary objects, and then load the rest in
the background. We do this with threads. We load the few objects we need from
the start, then create a new thread to load the rest of the objects, and in the
mean time, we create the interface and draw the map.

The same goes for the first run. The user does not need to wait for the program
to finish serializing and saving to files. By using threads, we can create the
data structures, and then immediately show the window to the user, while saving
the objects to files in the background.

During load time (when loading from the serialized files, after the window is
shown to the user) not all quadtrees are loaded. So we did encounter a problem
when querying the quadtrees, when not all of them were loaded, as our code
would try to use quadtrees not yet loaded. We solved this by putting the
querying in a try/catch, and when a problem occurred (index out of bounds, when
trying to access a quadtree not yet loaded), we simply stopped looking through
more quadtrees and just return what we found so far. Then later on, when all
quadtrees were loaded, we could return all edges. The user would not notice the
lack of roads, as the user only sees a limited amount of roads anyway.

We did encounter another problem, when trying to use the graph (for pathfinding)
when the graph was not done loading. We solved this by using a simple loop, that
check if the graph was set. As long as the graph is not available, the main
thread would simply hold and wait for the loading to finish. This is probably
the only time where the user would notice that everything is not quite loaded,
but the loading happens so fast, that it probably will not be a problem anyway.

With the nature of threads in mind, we cannot guarantee this behaviour, but no
problems should arise if the loading occurs in the wrong order.
