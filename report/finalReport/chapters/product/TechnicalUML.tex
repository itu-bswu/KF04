\chapter{UML-diagrams}
\label{UML}

Figure \ref{control-flow} on page \pageref{control-flow} shows our
implementation of the MVC architecture. Because we have exactly one window, we chose to name our view and controller ``\class{View}'' and 
``\class{Control}'', respectively. The same goes for the models. We only have one data 
source (the provided data set), and because of that our model is called
``\class{Model}''. We could have named these three classes something different that may have been more meaningful 
in terms of the \class{Map of Denmark} context, but we chose these name in order to make 
our architecture clear.

We wanted to keep the classes as ``trimmed'' as possible, although our
\class{Model} is quite long. But the amount of (public) methods is small, so
looking at it from outside, it is a skinny architecture. The reason we wanted to
keep the classes skinny, is because it is not the models responsibility to
deliver the same data in different ways, do a lot of calculations or provide
similar functionality. It only acts like a ``middle-man'', delivering data to
other classes. If some class wants the data in another way, they will need to convert it themselves.

\class{Control} makes sure the models and the view can understand each other. The view 
only knows about pixels, but it has no idea about UTM coordinates. The model only knows 
about UTM coordinates, but does not know anything about pixels. So for getting
these two to communicate, we need to convert pixels to UTM and vice versa.

We created some helper-classes (\class{PointMethods} and \class{RectangleMethods}), 
which are located in the \class{utils} package. These take care of checking whether a point 
is within the maximal bounds of the map, converting a pixel coordinate to a UTM coordinate 
and vice versa. We did this for being able to do this in several files, without the need to have 
duplicate code. For our program right now, this is not really a problem, as we
only have one model, one controller and a view. But if we were to have more, we would either need to copy 
these helper-methods into the other classes (BAD), or put them in a separate class. But even 
though we only have one of each, the helper-classes are still an advantage, as
they make the code cleaner and easier to maintain and test, because we can test
these helper-classes seperately.

So in essence, we have two parts (the model and the view) that need to communicate 
somehow, in order to display the data on the screen. But they speak different languages, so 
we put in a middle-man (the controller), responsible for the communication between the two.

\section{Control flow}
\label{UML-CF}
The easiest way to understand our flow through the program and how the individual parts 
talk together, is by using an example. Let us say the user has already clicked the map and 
placed a marker. Now the user clicks on the map again to place another marker.

\begin{figure}[h!]
\centering
\includegraphics[width=0.75\linewidth]{images/SimpleControlFlow}
\caption{The basic control flow of our Map program}
\label{control-flow}
\end{figure}

The controller has placed listeners in the view, so when a MouseClicked event
happens, the controller is called. First it checks if there is another marker
at the spot of the click. If there is, this will be removed, and the model is
told to clear the route. If there is still over two markers placed, the model is
asked to calculate a new route.

If there is not a marker where the user clicks, we place a new marker. If the
user has placed two or more pins, the controller calls its own findPath-method from point 1 to point 2, point 2 to 
point 3 and so on. The findPath()-method tells the model to find a path between the two 
points given.

The model then asks a helper-class to find a path, using the A* algorithm, and provides it 
with the graph and the two points. When a path is found, it is saved in the model, ready 
for use in the controller.

The final step is getting the view to draw the route. The controller gets ready for a repaint
by fetching the route from the model. Then it passes this route to the view's repaint-method, 
and the view paints the road.