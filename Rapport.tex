\documentclass[11pt,a4paper,titlepage]{article}
\usepackage[latin1]{inputenc}

\title{KF04 project report}
\author{
	Group 12:\\
	Jakob Melnyk\\
	Niklas Hansen\\
	Emil Juul Jacobsen\\
	Filip Hjermind Jensen\\
	Jens Dahl M�llerh�j\\
}

\date{April 5, 2011}
\begin{document}
\maketitle
\tableofcontents

\setlength{\parindent}{0pt}
\setlength{\parskip}{1.8ex plus 0.5ex minus 0.2ex}

\clearpage

\section{Features}
This section describes the features we have chosen to include and not to include
in our Map Of Denmark project for Spring 2011.
\begin{itemize}
  \item 
	Mousezoom scales with current window ratio
  \item 
	Move buttons on GUI for navigation
  \item
  	Zoom in and out functions on GUI 
  \item 
  	Roadnames shown at the bottom of the GUI when moused over
  \item 
  	Makes the map bigger when borders of GUI is dragged.
  \item
  	Reset zoom level using Escape button.
  \item
  	Shows less roads when zoomed far out
\end{itemize}
\subsection{Mousezoom}
	It is possible to zoom in on a specific part of the map by dragging a box
	around the area to be zoomed in on. The area shown after the zoom will be fixed
	to the ratio of the map. 
	
	This means that it will always show the part of the map that was selected with
	the mouse, but it will sometimes show more in either the width or the height
	depending on how the rectangle was drawn.
\subsection{Move buttons}
	It is possible to navigate the map by using the buttons on the
	graphical user interface. When the buttons are pressed, the view moves in the
	direction indicated by the arrow.
\subsection{Zoom in/out functions}
	It is possible to zoom in and out using the + and - buttons on the graphical
	user interface. It will do the zoom in/out on the center of the map.
\subsection{Reset zoom level}
	It is possible to reset the zoom level to what it was when the application
	started. This is done by pressing escape. This makes it much easier to navigate
	the map if the user has zoomed in a few times.
\subsection{Roadnames on mouseover}
	When mousing over a road on the map, the name of the road is shown at the
	bottom of the graphical user interface.
\subsection{Draggable borders}
\subsection{Less roads when zoomed far out}
\subsection{Not included}
	Currently the only way to navigate around the map is to use the buttons on the
	graphical user interface. We thought about both including a way to move around
	the map by either dragging with the mouse and/or using the arrow keys on the
	keyboard. However, we put it aside - for now - as a ``nice, but not
	must-have''-feature.
	
	We have chosen not to let the user select what road types he/she wants to be
	drawn on the map. We do not feel that this is necessary, because we only draw
	the bigger roads when zoomed out.
\section{Implementation}
	This section describes what choices we have made when implementing our Map Of
	Denmark project for Spring 2011.
\subsection{Model-View-Control}
\subsection{PixelToUTM}
\subsection{Control}
\subsection{Map}
\subsection{Roads}
\subsection{View}

\end{document}