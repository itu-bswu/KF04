\documentclass[11pt,a4paper,titlepage]{article}

\title{KF04 project report}
\author{
	Group 12:\\
	Jakob Melnyk\\
	Niklas Hansen\\
	Emil Juul Jacobsen\\
	Filip Hjermind Jensen\\
	Jens Dahl Moellerhoej\\
}

\date{April 5, 2011}
\begin{document}
\maketitle
\tableofcontents

\setlength{\parindent}{0pt}
\setlength{\parskip}{1.8ex plus 0.5ex minus 0.2ex}

\clearpage



\section{Introduction}
\section{Features}
\begin{itemize}
  \item 
	Mousezoom scales with current window ratio
  \item 
	Move buttons on GUI for navigation
  \item
  	Zoom in and out functions on GUI 
  \item 
  	Roadnames shown at the bottom of the GUI when moused over
  \item 
  	Makes the map bigger when borders of GUI is dragged.
  \item
  	 Reset zoom level using Escape button.
\end{itemize}

\subsection{Mousezoom}

\subsection{Move buttons}

\subsection{Zoom in/out functions}

\subsection{Roadnames on mouseover}

\subsection{Draggable borders}

\subsection{Reset zoom level}

\section{Implementation}

\subsection{Model-View-Control}
Skrives af Niklas

\subsection{Control}
Skrives af Jakob

\subsection{PixelToUTM}
Skrives af Jakob

\subsection{Map}
Map klassen st�r for den del af kortet der vises p� sk�rmen. Den holder feltet <bounds> der er den rektangel af kortet der vises i �jeblikket. N�r mapklassen skabes beregner den <bounds> til det mindste rektangel der viser hele kortet.
N�r View skal tegne kortet skaber map <Line> objekter af de  <krakEdges> som skal vises. Map s�rger for at lines har den rigtige farve og tykkelse.

\subsection{QuadTree og selektering af veje}
Vi har optimeret vores kort med et quadtree. Hver gang vi har brug for data om kortet, sender vi quadtr�et et rektangel med det omr�de vi �nsker data om. 
For at optimere visningen vores kort har vi valgt kun at vise nogle vejtyper. N�r man kigger p� hele Danmark er det ikke interessant at f� vist de enkelte sm�veje, og derfor er der ingen grund til at tegne dem. Samtidig optimere l�sningen kortet betydeligt.
Implementationen af denne l�sning spiller i meget h�j grad sammen med implementationen af QuadTree. Inden vi implementerede selekteringen overvejede vi to forskellige m�der at f� det til at spille sammen med QuadTree p�.
P� den ene side kunne vi allerede i quadtr�ets rod splitte tr�et op i flere dele, s�ledes at hver dels blade var af en bestemt vejtype.
P� den anden side kunne vi lader vejetyperne blive opdelt efter dybten i tr�et, s�ledes at de store veje i f�rste omgang blev fordelt i de f�rste knuder, og de mindre vejtyper senere blev fordelt ned igennem tr�et.
De to metoder havde hver deres fordele og ulemper. Den f�rste ville kr�ve flere node objekter, hvilket ville koste i RAM forbrug. Til geng�ld ville alle vejene i denne metode blive lagt i tr�ets blade.
Antallet af veje i den enkelte gruppering, samt antallet af grupperinger har v�ret til diskussion. P� den ene side vil et stort antal veje ved det enkelte zoom niveau g�r det lettere at genkende omr�det. P� den anden side vil det g� ud over performance.
Vi har besluttet at vise en stor del af vejene, da vi ikke har omr�dedata til at tegne hvad der er vand og hvad der er land.

\subsection{Lines}
Skrives af Filip

\subsection{View}
Skrives af Emil

\end{document}