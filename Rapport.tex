\documentclass[11pt,a4paper,titlepage]{article}
\usepackage[latin1]{inputenc}

\title{KF04 project report}
\author{
	Group 12:\\
	Jakob Melnyk\\
	Niklas Hansen\\
	Emil Juul Jacobsen\\
	Filip Hjermind Jensen\\
	Jens Dahl M�llerh�j\\
}

\date{April 5, 2011}
\begin{document}
\maketitle
\tableofcontents

\setlength{\parindent}{0pt}
\setlength{\parskip}{1.8ex plus 0.5ex minus 0.2ex}

\clearpage

\section{Features}
This section describes the features we have chosen to include and not to include
in our Map Of Denmark project for Spring 2011.
\begin{itemize}
  \item 
	Mousezoom scales with current window ratio
  \item 
	Move buttons on GUI for navigation
  \item
  	Zoom in and out functions on GUI 
  \item 
  	Roadnames shown at the bottom of the GUI when moused over
  \item 
  	Makes the map bigger when borders of GUI is dragged.
  \item
  	Reset zoom level using Escape button.
  \item
  	Shows less roads when zoomed far out
\end{itemize}
\subsection{Mousezoom}
	It is possible to zoom in on a specific part of the map by dragging a box
	around the area to be zoomed in on. The area shown after the zoom will be fixed
	to the ratio of the map. 
	
	This means that it will always show the part of the map that was selected with
	the mouse, but it will sometimes show more in either the width or the height
	depending on how the rectangle was drawn.
\subsection{Move buttons}
	It is possible to navigate the map by using the buttons on the
	graphical user interface. When the buttons are pressed, the view moves in the
	direction indicated by the arrow.
\subsection{Zoom in/out functions}
	It is possible to zoom in and out using the + and - buttons on the graphical
	user interface. It will do the zoom in/out on the center of the map.
\subsection{Reset zoom level}
	It is possible to reset the zoom level to what it was when the application
	started. This is done by pressing escape. This makes it much easier to navigate
	the map if the user has zoomed in a few times.
\subsection{Roadnames on mouseover}
	When mousing over a road on the map, the name of the road is shown at the
	bottom of the graphical user interface.
\subsection{Draggable borders}
	When dragging the window of the application, the borders of the map will
	increase in width and/or height. This means that if the border(s) is dragged,
	the view will show more of the map.
\subsection{Less roads when zoomed far out}
	The further the user has zoomed out on the map, the less roads - and thus
	details - are shown on the map. The lower level of detail is not a problem when
	zoomed far out, but the map can be navigated and zoomed in and out faster.
\subsection{Not included}
	Currently the only way to navigate around the map is to use the buttons on the
	graphical user interface. We thought about both including a way to move around
	the map by either dragging with the mouse and/or using the arrow keys on the
	keyboard. However, we put it aside - for now - as a ``nice, but not
	must-have''-feature.
	
	We have chosen not to let the user select what road types he/she wants to be
	drawn on the map. We do not feel that this is necessary, because we only draw
	the bigger roads when zoomed out.
\pagebreak
\section{Implementation}
	This section describes what choices we have made when implementing our Map Of
	Denmark project for Spring 2011.
\subsection{Model-View-Control}
Skrives af Niklas

\subsection{Control}
Skrives af Jakob

\subsection{PixelToUTM}
Skrives af Jakob

\subsection{Map}
	Map klassen st�r for den del af kortet der vises p� sk�rmen. Den holder feltet
	<bounds> der er den rektangel af kortet der vises i �jeblikket. N�r mapklassen
	skabes beregner den <bounds> til det mindste rektangel der viser hele kortet.
	N�r View skal tegne kortet skaber map <Line> objekter af de  <krakEdges> som
	skal vises. Map s�rger for at lines har den rigtige farve og tykkelse.
\subsection{QuadTree og selektering af veje}
	Vi har optimeret vores kort med et quadtree. Hver gang vi har brug for data om
	kortet, sender vi quadtr�et et rektangel med det omr�de vi �nsker data om. For
	at optimere visningen vores kort har vi valgt kun at vise nogle vejtyper. N�r
	man kigger p� hele Danmark er det ikke interessant at f� vist de enkelte
	sm�veje, og derfor er der ingen grund til at tegne dem. Samtidig optimerer
	l�sningen kortet betydeligt. 
	
	Implementationen af denne l�sning spiller i meget h�j grad sammen med
	implementationen af QuadTree. Inden vi implementerede selekteringen overvejede
	vi to forskellige m�der at f� det til at spille sammen med QuadTree p�. P� den
	ene side kunne vi allerede i quadtr�ets rod splitte tr�et op i flere dele,
	s�ledes at hver dels blade var af en bestemt vejtype. P� den anden side kunne
	vi lader vejetyperne blive opdelt efter dybten i tr�et, s�ledes at de store
	veje i f�rste omgang blev fordelt i de f�rste knuder, og de mindre vejtyper
	senere blev fordelt ned igennem tr�et. 

	De to metoder havde hver deres fordele og ulemper.
	Den f�rste ville kr�ve flere node objekter, hvilket ville koste i RAM forbrug.
	Til geng�ld ville alle vejene i denne metode blive lagt i tr�ets blade.
	Antallet af veje i den enkelte gruppering, samt antallet af grupperinger har
	v�ret til diskussion. P� den ene side vil et stort antal veje ved det enkelte
	zoom niveau g�r det lettere at genkende omr�det. P� den anden side vil det g�
	ud over performance. 

	Vi har besluttet at vise en stor del af vejene, da vi ikke har omr�dedata til
	at tegne hvad der er vand og hvad der er land.
\subsection{Lines}
	Line klassen benyttes ifbm. at KrakEdge objekterne skal have et start- og et
	slutpunkt som er relative til sk�rmen. Derudover indeholder Line klassen
	information omkring tykkelse og farve p� den p�g�ldende vej. Tykkelsen og
	farven kunne vi v�lge at specificere i enten Map klassen eller i selve Line
	klassen. Vi valgte at g�re det i Map for at spare RAM.
\subsection{View}
	Skrives af Emil
\end{document}